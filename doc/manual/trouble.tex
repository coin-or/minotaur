%% Example of a Handler: QGHandler

\section{Troubleshooting}
\begin{codeenv}
 CMake Warning (dev) at /usr/share/cmake-2.8/Modules/FortranCInterface/Detect.cmake:93 (list):
   Policy CMP0007 is not set: list command no longer ignores empty elements.
   Run "cmake --help-policy CMP0007" for policy details.  Use the cmake_policy
   command to set the policy and suppress this warning.  List has value =
   [;my_sub;_].
 Call Stack (most recent call first):
   /usr/share/cmake-2.8/Modules/FortranCInterface.cmake:118 (include)
   CMakeLists.txt:15 (INCLUDE)
 This warning is for project developers.  Use -Wno-dev to suppress it.
\end{codeenv}
This an issue with some versions of CMake (e.g. 2.8.0). It can be safely
ignored.

\begin{codeenv}
src/testing/LapackUT.cpp:27: error: variable or field ‘F77_FUNC’ declared void
\end{codeenv}
This an issue with some versions of CMake (e.g. 2.8.0). It can be mitigated by
switching to version 2.8.1.

\begin{codeenv}
error: variable or field `F77_FUNC' declared void
error: `dpotrf' was not declared in this scope
error: `DPOTRF' was not declared in this scope
\end{codeenv}
Then CMake detects a fortran compiler, but can not
detect its mangling scheme correctly.


\begin{codeenv}
member: .libs/libmntrbqpd.a(BqpdAux.o) cputype (7) does not 
match previous archive members cputype (16777223) 
(all members must match)
\end{codeenv}

It means that the fortran compiler and C++ compilers built objects for different
architectures. Sometimes the Fortran compiler builds 32-bit code on a 64-bit
machine. Try appending \code+FFLAGS=-m64+ to the arguments of configure. Then do 
\code+make clean+, and then \code+make install+.

\begin{codeenv}
Undefined symbols:
  "_dsyevr_", referenced from:
      Minotaur::EigenCalculator::calculate_()      in Eigen.o
      Minotaur::EigenCalculator::calculate_()      in Eigen.o
\end{codeenv}
\code+BUILD_SHARED_LIBS+ should be \code+OFF+ in ccmake-menu.

\begin{codeenv}
-lcrt0.a  not found
\end{codeenv}
Seen only on a Mac, when building examples. Remove ``-static'' flag from the
Makefile.


Be careful about control characters (or spaces) that may be entered by cmake.
It may often be useful to clear all cached values for variables by doing the equivalent of 


\begin{codeenv}
find . -name "*Cache*" -exec rm {} \;
\end{codeenv}
and then ccmake ..
or your cmake command
